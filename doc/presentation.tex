\documentclass{beamer}
\mode<presentation>

%% packages
\usepackage{beamerthemeshadow}
\usepackage{xcolor}
\usepackage[english]{babel}
\usepackage[latin1]{inputenc}
\usepackage{times}
\usepackage[T1]{fontenc}
\usepackage{wrapfig}
\usepackage{color}
\usepackage{graphicx}
\usepackage{subfigure}
\usepackage{multirow}
\usepackage{amsmath}
\usepackage{tikz}
\usetikzlibrary{automata,positioning,arrows}
\usepackage{tikz,ifthen}
\usepackage{etoolbox}
\usepackage{hyperref}

%% The Beamer class comes with a number of default slide themes which change the colors and layouts of slides. For more details go to  http://deic.uab.es/~iblanes/beamer_gallery/
%\usetheme{default}
%\usetheme{AnnArbor} % Yellow colored theme, 3 parition in footer
%\usetheme{Antibes}  % Blue shadowed theme, 2 partition in footer
%\usetheme{Bergen}   % left side navigation, 2 partition in footer
%\usetheme{Berkeley}
%\usetheme{Berlin}   % bottom vertical partition, horizontal top navigation
%\usetheme{Boadilla} % light blue navigation, 3 partition at bottom
%\usetheme{CambridgeUS} % red color cool theme, 3 partition
%\usetheme{Copenhagen}
%\usetheme{Darmstadt}
%\usetheme{Dresden}
\usetheme{EastLansing} % cool green colored theme
%\usetheme{Frankfurt}
%\usetheme{Goettingen}
%\usetheme{Hannover}
%\usetheme{Ilmenau}
%\usetheme{JuanLesPins}
%\usetheme{Luebeck}
%\usetheme{Madrid}
%\usetheme{Malmoe}
%\usetheme{Marburg}
%\usetheme{Montpellier}
%\usetheme{PaloAlto}
%\usetheme{Pittsburgh}
%\usetheme{Rochester}
%\usetheme{Singapore}
%\usetheme{Szeged}
%\usetheme{Warsaw}

%% As well as themes, the Beamer class has a number of color themes for any slide theme.
%\usecolortheme{albatross}
%\usecolortheme{beaver}
%\usecolortheme{beetle}
%\usecolortheme{crane}
%\usecolortheme{dolphin}
%\usecolortheme{dove}
%\usecolortheme{fly}
%\usecolortheme{lily}
%\usecolortheme{orchid}
%\usecolortheme{rose}
%\usecolortheme{seagull}
%\usecolortheme{seahorse}
%\usecolortheme{whale}
%\usecolortheme{wolverine}

%% font themes
%\usefonttheme{structurebold}
%\usefonttheme{professionalfonts}
%\usefonttheme{structuresmallcapsserif}
%\usefonttheme{serif}
\usefonttheme{structureitalicserif}

%% color definition
\definecolor{theme_color}{RGB}{0,51,25}

%% custom color
%\setbeamercolor{alerted text}{fg=orange}
%\setbeamercolor{background canvas}{bg=white}
%\setbeamercolor{block body alerted}{bg=normal text.bg!90!black}
%\setbeamercolor{block body}{bg=normal text.bg!90!black}
%\setbeamercolor{block body example}{bg=normal text.bg!90!black}
%\setbeamercolor{block title alerted}{use={normal text,alerted text},fg=alerted text.fg!75!normal text.fg,bg=normal text.bg!75!black}
%\setbeamercolor{block title}{bg=blue}
%\setbeamercolor{block title example}{use={normal text,example text},fg=example text.fg!75!normal text.fg,bg=normal text.bg!75!black}
%\setbeamercolor{fine separation line}{}
%\setbeamercolor{frametitle}{fg=brown}
%\setbeamercolor{item projected}{fg=black}
%\setbeamercolor{normal text}{bg=black,fg=yellow}
%\setbeamercolor{palette sidebar primary}{use=normal text,fg=normal text.fg}
%\setbeamercolor{palette sidebar quaternary}{use=structure,fg=structure.fg}
%\setbeamercolor{palette sidebar secondary}{use=structure,fg=structure.fg}
%\setbeamercolor{palette sidebar tertiary}{use=normal text,fg=normal text.fg}
%\setbeamercolor{section in sidebar}{fg=brown}
%\setbeamercolor{section in sidebar shaded}{fg= grey}
%\setbeamercolor{separation line}{}
%\setbeamercolor{sidebar}{bg=red}
%\setbeamercolor{sidebar}{parent=palette primary}
\setbeamercolor{structure}{fg=theme_color}
%\setbeamercolor{subsection in sidebar}{fg=brown}
%\setbeamercolor{subsection in sidebar shaded}{fg= grey}
%\setbeamercolor{title}{fg=brown}
%\setbeamercolor{titlelike}{fg=brown}

%% control header
\setbeamertemplate{headline}{}

%% control footer
%\setbeamertemplate{footline}
%\setbeamertemplate{footline}[page number]

%% removes navigation symbol
%\setbeamertemplate{navigation symbols}{}

%% bibilography settings
\setbeamertemplate{frametitle continuation}[from second]
\setbeamercolor*{bibliography entry title}{fg=black}
\setbeamercolor*{bibliography entry author}{fg=black}
\setbeamercolor*{bibliography entry location}{fg=black}
\setbeamercolor*{bibliography entry note}{fg=black}

%% same slide number for continuation slides
\newcounter{multipleslide}
\makeatletter
\newcommand{\multipleframe}
{
	\setcounter{multipleslide}{\value{framenumber}}
	\stepcounter{multipleslide}
	\patchcmd{\beamer@@tmpl@footline}
	{\insertframenumber}
	{\themultipleslide}
	{}
	{}
}
\newcommand{\restoreframe}
{
	\patchcmd{\beamer@@tmpl@footline}
	{\themultipleslide}
  	{\insertframenumber}
	{}
	{}
	\setcounter{framenumber}{\value{multipleslide}}
}
\makeatother

%% title page
\title[Dynamic Server Consolidation (simcon)]
{
	R \& D Project (CS 692) \\
	Dynamic Server Consolidation Problem \\
	as Markov Decision Process \\
	(New Approaches)
}
\author[Aman Mangal, (IIT Bombay)]
{	\sffamily
	\textbf{By:} Aman Mangal \\
	\textbf{Advisor:} Prof Varsha Apte \\
	IIT Bombay
}
\date{\today}

\begin{document}
\begin{frame}
\addtocounter{framenumber}{-1}
\begin{figure}
    \includegraphics[width=2.3cm]{images/iitb}
\end{figure}
\titlepage
\end{frame}

\begin{frame}\frametitle{Presentation Outline}
\tableofcontents
\end{frame}

\section{Introduction}
\begin{frame}{Introduction}
\addtocounter{framenumber}{-1}
\tableofcontents[currentsection]
\end{frame}

\subsection{Current Status}
\begin{frame}{What We Have Done?}
\begin{itemize}
\item Proposed a utilization based profit model for a data center-
\begin{itemize}
\item Profit = Fee/Revenue - SLA Penalty - Power Cost
\item Resilient to change in technology, independent of the assumption that migration is always costly
\end{itemize}
\item Formulation of DSCP as Finite Horizon MDP for cyclic workload-
\begin{itemize}
\item Deterministic actions, fixed number of phases
\item Long term profit based, two step solution
\item MDP solution using backward induction method proposed by Puterman (Dynamic Programming)
\end{itemize}
\end{itemize}
\end{frame}

\subsection{Issues \& Direction of Improvements}
\multipleframe
\begin{frame}{Issues}
\begin{itemize}
\item Profit model does not fully mimic the response time of the system
\begin{itemize}
\item Anshu's work
\end{itemize}
\item Has exponential asymptotic complexity in the number of VMs
\begin{itemize}
\item B  $< \left(\frac{0.79M}{\log M}\right)^{M}$
\item Order of B: O$\left(\frac{M}{\log M}\right) ^{M}$
\item Time Complexity: \\
\hspace*{1cm} $O\left(N_P*B^{2}*M! + \left(N_P-1\right)*B^{3}\right)$
\item Space Complexity : \\
\hspace*{1cm} $O\left(N_P*B^{2}*M + N_P*B^{2}\right)$
\end{itemize}
\end{itemize}
\end{frame}

\begin{frame}{Possible Approaches to Reduce Complexity}
\begin{itemize}
\item Step 1 (local optimization step)
\item Step 2 (Global optimization step)
\begin{itemize}
\item Shortest path algorithms
\begin{itemize}
\item Single source (Dijkstra's Algorithm)
\item Single pair (A* search)
\item All pairs
\end{itemize}
\item MDP specific solutions
\begin{itemize}
\item Variation of policy iteration and evaluation
\item Papers related to solving large MDPs
\end{itemize}
\item Reinforcement learning based solutions
\begin{itemize}
\item Approximate solutions
\item UCT and its variations
\end{itemize}
\item Combination of offline and online solutions
\end{itemize}
\end{itemize}
\end{frame}
\restoreframe

\section{Shortest Path Algorithms}
\begin{frame}{Shortest Path Algorithms}
\addtocounter{framenumber}{-1}
\tableofcontents[currentsection]
\end{frame}

\subsection{Applicability \& Problem Statement}
\begin{frame}{Problem Statement}
\begin{itemize}
\item Given a weighted directed graph G = (V, E), find the shortest path i.e. path with lowest weight sum
\begin{center}
\begin{figure}[h]
\begin{tikzpicture}[scale=0.6, transform shape, shorten >=1pt, node distance=2cm,auto]
\tikzstyle{every state}=[draw=blue!50, very thick, fill=blue!20]
\node[state] (s11)                {$s_{1,1}$};
\node[state] (s12) [right of=s11] {$s_{1,2}$};
\node[state] (s22) [below of=s12] {$s_{2,2}$};
\node[state] (s32) [below of=s22] {$s_{3,2}$};
\node[state] (s13) [right of=s12] {$s_{1,3}$};
\node[state] (s23) [below of=s13] {$s_{2,3}$};
\node[state] (s33) [below of=s23] {$s_{3,3}$};
\node[state] (s14) [right of=s13] {$s_{1,4}$};
\node[state] (s24) [below of=s14] {$s_{2,4}$};
\node[state] (s34) [below of=s24] {$s_{3,4}$};

\tikzstyle{every state}=[draw=none, very thick, node distance=1cm]
\node[state] (sd2) [below of=s32] {$\vdots$};
\node[state] (sd3) [below of=s33] {$\vdots$};
\node[state] (sd4) [below of=s34] {$\vdots$};

\node[state] (s1d) [right of=s14] {$\cdots$};
\node[state] (s2d) [right of=s24] {$\cdots$};
\node[state] (s3d) [right of=s34] {$\cdots$};
\node[state] (sdd) [right of=sd4] {$\ddots$};
\node[state] (sdn) [below of=sdd] {$\cdots$};

\node[state] (snd) [right of=sdd] {$\vdots$};

\tikzstyle{every state}=[draw=blue!50, very thick, fill=blue!20, node distance=1cm, auto]
\node[state] (sn2) [below of=sd2] {$s_{B,2}$};
\node[state] (sn3) [below of=sd3] {$s_{B,3}$};
\node[state] (sn4) [below of=sd4] {$s_{B,4}$};

\node[state] (s1n) [right of=s1d] {$s_{1,N_p}$};
\node[state] (s2n) [right of=s2d] {$s_{2,N_p}$};
\node[state] (s3n) [right of=s3d] {$s_{3,N_p}$};
\node[state] (snn) [right of=snd] {$s_{B,N_p}$};

\tikzstyle{every state}=[draw=blue!50, very thick, fill=blue!20, node distance=2cm, auto]
\node[state] (s11') [right of=s1n] {$s_{1,1}$};

\foreach \x in {s12, s22, s32, sn2}
{\path[->] (s11) edge node {} (\x);}
\foreach \x in {s12, s22, s32, sn2}
{\path[->] (\x) edge node {} (s13)
                edge node {} (s23)
                edge node {} (s33)
                edge node {} (sn3);}
\foreach \x in {s13, s23, s33, sn3}
{\path[->] (\x) edge node {} (s14)
                edge node {} (s24)
                edge node {} (s34)
                edge node {} (sn4);}
\foreach \x in {s1n, s2n, s3n, snn}
{\path[->] (\x) edge node {} (s11');}
\end{tikzpicture}
\end{figure}
\end{center}
\item \#E = $(N_p-2)*B^2 + 2*B$
\item \#V = $(N_p-1)*B + 2$
\end{itemize}
\end{frame}

\begin{frame}{Applicability}
\begin{itemize}
\item Requires absence of non-positive total weight cycles in the graph
\begin{itemize}
\item True in case of our formulation of DSCP as MDP
\item There are no cycles in the MDP\footnote{From now onward MDP refers to our formulation of DSCP as MDP unless specified explicitly}
\end{itemize}
\item Dijkstra's algorithm requires non-negative edges
\begin{itemize}
\item Shifting the edge weights by a constant amount will work in MDP
\item Let's prove it but before that let's formalize the problem we are trying to solve
\end{itemize}
\end{itemize}
\end{frame}

\multipleframe
\begin{frame}{Negative Edges}
\begin{itemize}
\item Consider 2 version of the same problem
\item Graph B having more weights by a fixed offset than the graph A such that all the weights are positive in graph B
\end{itemize}
\begin{columns}[onlytextwidth]
\begin{column}{0.5\textwidth}
\begin{figure}[h]
\begin{center}
\begin{tikzpicture}[scale=0.6, transform shape, shorten >=1pt, node distance=2cm,auto]
\tikzstyle{every state}=[draw=blue!50, very thick, fill=blue!20]
\node[state] (s11)                {$s_{1,1}$};
\node[state] (s12) [right of=s11] {$s_{1,2}$};
\node[state] (s22) [below of=s12] {$s_{2,2}$};
\node[state] (s32) [below of=s22] {$s_{3,2}$};
\node[state] (s13) [right of=s12] {$s_{1,3}$};
\node[state] (s23) [below of=s13] {$s_{2,3}$};
\node[state] (s33) [below of=s23] {$s_{3,3}$};

\tikzstyle{every state}=[draw=none, very thick, node distance=1.4cm]
\node[state] (sd2) [below of=s32] {$\vdots$};
\node[state] (sd3) [below of=s33] {$\vdots$};

\node[state] (s1d) [right of=s13] {$\cdots$};
\node[state] (s2d) [right of=s23] {$\cdots$};
\node[state] (s3d) [right of=s33] {$\cdots$};
\node[state] (sdd) [right of=sd3] {$\ddots$};
\node[state] (sdn) [below of=sdd] {$\cdots$};

\node[state] (snd) [right of=sdd] {$\vdots$};

\tikzstyle{every state}=[draw=blue!50, very thick, fill=blue!20, node distance=1.4cm, auto]
\node[state] (sn2) [below of=sd2] {$s_{B,2}$};
\node[state] (sn3) [below of=sd3] {$s_{B,3}$};

\node[state] (s1n) [right of=s1d] {$s_{1,N_p}$};
\node[state] (s2n) [right of=s2d] {$s_{2,N_p}$};
\node[state] (s3n) [right of=s3d] {$s_{3,N_p}$};
\node[state] (snn) [right of=sdn] {$s_{B,N_p}$};

\tikzstyle{every state}=[draw=blue!50, very thick, fill=blue!20, node distance=2cm, auto]
\node[state] (s11') [right of=s1n] {$s_{1,1}$};

\path[->] (s11) edge node {} (s22);
\path[->] (s22) edge node {} (s23);
\path[->] (s23) edge node {} (s3d);
\path[->] (s3d) edge node {} (s3n);
\path[->] (s3n) edge node {} (s11');
\end{tikzpicture}
\end{center}
\caption{(A)}
\end{figure}
\end{column}
\vrule{}
\begin{column}{0.5\textwidth}
\begin{figure}[h]
\begin{center}
\begin{tikzpicture}[scale=0.6, transform shape, shorten >=1pt, node distance=2cm,auto]
\tikzstyle{every state}=[draw=blue!50, very thick, fill=blue!20]
\node[state] (s11)                {$s_{1,1}$};
\node[state] (s12) [right of=s11] {$s_{1,2}$};
\node[state] (s22) [below of=s12] {$s_{2,2}$};
\node[state] (s32) [below of=s22] {$s_{3,2}$};
\node[state] (s13) [right of=s12] {$s_{1,3}$};
\node[state] (s23) [below of=s13] {$s_{2,3}$};
\node[state] (s33) [below of=s23] {$s_{3,3}$};

\tikzstyle{every state}=[draw=none, very thick, node distance=1.4cm]
\node[state] (sd2) [below of=s32] {$\vdots$};
\node[state] (sd3) [below of=s33] {$\vdots$};

\node[state] (s1d) [right of=s13] {$\cdots$};
\node[state] (s2d) [right of=s23] {$\cdots$};
\node[state] (s3d) [right of=s33] {$\cdots$};
\node[state] (sdd) [right of=sd3] {$\ddots$};
\node[state] (sdn) [below of=sdd] {$\cdots$};

\node[state] (snd) [right of=sdd] {$\vdots$};

\tikzstyle{every state}=[draw=blue!50, very thick, fill=blue!20, node distance=1.4cm, auto]
\node[state] (sn2) [below of=sd2] {$s_{B,2}$};
\node[state] (sn3) [below of=sd3] {$s_{B,3}$};

\node[state] (s1n) [right of=s1d] {$s_{1,N_p}$};
\node[state] (s2n) [right of=s2d] {$s_{2,N_p}$};
\node[state] (s3n) [right of=s3d] {$s_{3,N_p}$};
\node[state] (snn) [right of=sdn] {$s_{B,N_p}$};

\tikzstyle{every state}=[draw=blue!50, very thick, fill=blue!20, node distance=2cm, auto]
\node[state] (s11') [right of=s1n] {$s_{1,1}$};

\path[->] (s11) edge node {} (s12);
\path[->] (s12) edge node {} (s33);
\path[->] (s33) edge node {} (s3d);
\path[->] (s3d) edge node {} (s2n);
\path[->] (s2n) edge node {} (s11');
\end{tikzpicture}
\end{center}
\caption{(B)}
\end{figure}
\end{column}
\end{columns}
\end{frame}

\begin{frame}{Negative Edges}
\begin{itemize}
\item Proof by contradiction, assuming only one optimal path in a graph
\item Let's say optimal path in both the graphs is different, $P_A$ and $P_B$
\item Cost of $P_A$, in Graph B-\\
\hspace*{1cm} $CB_{P_A}$ = ($CA_{P_A}$ + $N_p$*offset)
\item Cost of optimal path of graph B, in Graph B-\\
\hspace*{1cm} $CB_{P_B}$ = ($CA_{P_B}$ + $N_p$*offset)
\item Clearly, $CA_{P_A} \leq CA_{P_B}  \Rightarrow  CB_{P_A} \leq CB_{P_B}$, leads to contradiction
\item This proof can be extended to graphs having more than one optimal path
\end{itemize}
\end{frame}
\restoreframe

\subsection{Complexity Analysis}
\begin{frame}{Complexity Analysis}
\begin{table}
\begin{tabular}{|p{2.5cm}|p{4cm}|c|}
\hline 
 & \textbf{Time Complexity} & \textbf{Space Complexity} \\ 
\hline 
Backward Induction & $N_P*B^{2}*M! + N_p*B^3$ & $N_P*B^{2}*M$ \\ 
\hline 
Dijkstra's Algorithm & $N_P*B^{2}*M! + N_p*B^3$ & $N_p*B^2$ \\ 
\hline 
Bellman-Ford Algorithm & $N_P*B^{2}*M! + N_p^2*B^4$ & $N_p*B^2$ \\ 
\hline 
A* search & $N_P*B^{2}*M! + N_p*B^3$ & $N_p*B^2$ \\ 
\hline 
\end{tabular} 
\end{table}
\begin{itemize}
\item In practice, complexity follows-
\begin{center}
Backward Induction > Dijkstra's Algorithm > A* search
\end{center}
\end{itemize}
\end{frame}

\section{References}
\multipleframe
\begin{frame}[allowframebreaks]
\frametitle{References}
\begin{thebibliography}{9}
\bibitem{permut}
    http://serverbob.3x.ro/IA/DDU0147.html
\bibitem{coursera}
	http://coursera.org
\end{thebibliography}
\end{frame}
\restoreframe

\begin{frame} \frametitle{Thank you}
\Huge{\centerline{\color{theme_color}Questions?}}
\end{frame}

\end{document}
